\documentclass[../portafolio.tex]{subfiles}

% Solo agregue paquetes en el preámbulo de ../portafolio.tex

\begin{document}

% En esta sección, explique en detalle los siguientes aspectos:
% - Fecha de realización de la actividad
% - Título de la actividad (dentro de \section)
% - Un párrafo explicando cuál es el objetivo de la actividad
% - Nombre de personas con quien trabajó en la actividad
% - Una selección de evidencias de que usted hizo esta actividad (imágenes, códigos, respuestas a un problema teórico, etc.)
% - Una conclusión breve (qué aprendió con la actividad, qué no entendió, qué faltó trabajar, qué recomienda para futuras sesiones)

% Numero máximo de palabras en esta sección: 1000 palabras.

%%%%%%%%%%%%%%%%%%%%%%%%%%%%%%%%%%%%%%%%%%%%%%%%%%%%%%%%%%%%%%%%%%%%%%%%%%%%%%%%
\section{Introducci\'on Git}   % ejemplo: Derivadas numéricas , introducción a git , 

\hfill \textbf{Fecha de la actividad:} 19 de agosto de 2022

\medskip

%---------------------------------------------------------------------------------
% Introducción/objetivos de la actividad
En esta clase lo que se hizo fue trabajar en la configuraci\'on de un repositorio personal, para la asignatura, en GitHub. Se configuro a modo de poder hacer un seguimiento clase a clase del progreso adem\'as de una forma de presentar evidencias de nuestro trabajo. Adem\'as, se ense\~no como trabajar sobre este repositorio de forma local ocupando el comando: 
	\begin{lstlisting}[language=bash] 
$ git clone git@github.com:fiscomp2-UdeC2022/portafolio-ayalin7.git
	\end{lstlisting} 
Que corresponde a la llave SSH de mi repositorio personal. Luego vimos los diversos comando que existen, y necesitamos aprender, entre ellos:

\begin{lstlisting}[language=bash]
$ git push 
$ git pull
\end{lstlisting}
Se ense\~n\'o tambien las implicancias del software Git, c\'omo se utliza en \'area de trabajo colaborativo, la creaci\'on de repositorios y la aplicaci\'on de estos en la asignatura.
%---------------------------------------------------------------------------------
% Con quién hizo esta actividad
Realic\'e esta actividad de manera individual, considerando que hay un paso a paso en el repositorio hecho por el profesor. 


%---------------------------------------------------------------------------------
% Selección de evidencias


\end{document}
