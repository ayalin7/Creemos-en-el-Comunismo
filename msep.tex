\documentclass[../portafolio.tex]{subfiles}

% Solo agregue paquetes en el preámbulo de ../portafolio.tex

\begin{document}

% En esta sección, explique en detalle los siguientes aspectos:
% - Fecha de realización de la actividad
% - Título de la actividad (dentro de \section)
% - Un párrafo explicando cuál es el objetivo de la actividad
% - Nombre de personas con quien trabajó en la actividad
% - Una selección de evidencias de que usted hizo esta actividad (imágenes, códigos, respuestas a un problema teórico, etc.)
% - Una conclusión breve (qué aprendió con la actividad, qué no entendió, qué faltó trabajar, qué recomienda para futuras sesiones)

% Numero máximo de palabras en esta sección: 1000 palabras.

%%%%%%%%%%%%%%%%%%%%%%%%%%%%%%%%%%%%%%%%%%%%%%%%%%%%%%%%%%%%%%%%%%%%%%%%%%%%%%%%
\section{Integración numérica básica}   % ejemplo: Derivadas numéricas , introducción a git , 

\hfill \textbf{Fecha de la actividad:} 02 de septiembre de 2022

\medskip

%---------------------------------------------------------------------------------
% Introducción/objetivos de la actividad
En este , la cual practicamos los conocimiento de integracion numerica,
en particular la regla del punto medio la cual usa el punto del medio entre los limites de integracion

%---------------------------------------------------------------------------------
% Con quién hizo esta actividad
Escriba un párrafo simple indicando si hizo la actividad
individualmente o en grupo. Si lo hizo en grupo, indique los nombres
de las personas con quien trabajó.


%---------------------------------------------------------------------------------
% Selección de evidencias

Se pide demostrar, 
\begin{equation}
\int_{a}^{b} dx f(x) = h f\left(\frac{a+b}{2}\right) + O(h^3) \notag
\end{equation}

con $h=b-a$ el largo del intervalo.

\medskip

Primero, consideraremos el punto medio entre a y b como $M = (a+b)/2$. Vemos que la expansión en series de Taylor en torno a $M$ de la función $f(x)$ es:

\begin{align}
   f(x)  & = f(M) + f'(M)(x-M) + \frac{1}{2} f''(M)(x-M)^{2} + \cdots  \\
         & = f(M) + f'(M)(x-M) + \frac{1}{2} f''(\xi)(x-M)^{2} \label{lol}
\end{align}

donde $M < \xi < x$. Ahora, al reemplazar $f(x)$ en (\ref{lol}) tenemos que,

\begin{align}
    \int_{a}^{b} dx f(x) & = \int_{a}^{b}dx f(M) + \int_{a}^{b}dx f'(M)(x-M) + \frac{1}{2} \int_{a}^{b}dx f''(M)(x-M)^{2}   \\
                         & = f(M)(b-a) + f'(M) \left[ \frac{(x-M)^{2}}{2} \right]_{\;a}^{\;b} + \frac{1}{2} \int_{a}^{b}dx f''(\xi)(x-M)^{2} \label{T}\\
                         & = f(M)(b-a) + f'(M) \left( \left( \frac{(b-M)^{2}}{2} \right) - \left( \frac{(a-M)^{2}}{2} \right) \right) + \frac{1}{2} \int_{a}^{b}dx f''(\xi)(x-M)^{2}  \\
                         & = f(M)(b-a) + \frac{1}{2} \int_{a}^{b}dx f''(\xi)(x-M)^{2}
\end{align}

donde, usando el teorema del valor medio, notamos que existe un valor $a < \eta < b$ tal que,

\begin{align}
   \int_{a}^{b}dx f(x) & = f(M)(b-a) + \frac{1}{2} f''(\eta) \int_{a}^{b}dx (x-M)^{2} \notag \\
                       & = f(M)(b-a) + \frac{1}{2} f''(\eta) \left[ \frac{(x-M)^{3}}{3} \right]_{\;a}^{\;b} \notag \\
                       %= f(M)(b-a) + \frac{1}{6} f''(\eta) \left( (b-M)^{3} - (a-M)^{3} \right)\\
                       %= f(M)(b-a) + \frac{1}{6} f''...
                       & = f(M)(b-a) + \frac{1}{6} f''(\eta) \left( \left( \frac{(b-a)^{3}}{8} \right) + \left( \frac{(b-a)^{3}}{8} \right) \right) \notag \\
                       & = f(M)(b-a) + \frac{1}{24} f''(\eta) h^{3} \label{error}
\end{align}

vemos que el ultimo termino de (\ref{error}) es un error de orden tres. Por lo que, la expresión final nos queda,

\begin{equation}
  \int_{a}^{b}dx f(x) = h f\left(\frac{a+b}{2}\right) + O(h^{3})
\end{equation}

\subsection{Una subsección}

Si cree necesario, puede separar el texto en subsecciones.



\end{document}
